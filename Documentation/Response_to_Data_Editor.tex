\documentclass[twoside,12pt,leqno]{article}
%\documentclass[twoside,12pt,leqno]{report}

\usepackage{amsmath,amssymb,amsfonts,amsthm,mathrsfs,multirow,upgreek}
\usepackage[capposition=bottom]{floatrow} % For Figure Notes
\usepackage{graphicx,pstricks,epstopdf}
\usepackage{url}      % This package helps to typeset urls

%\usepackage{apacite}
%\usepackage{natbib}   % This is a great aid with bibliographies
%\setcitestyle{authoryear,round,semicolon,aysep={},yysep={,},notesep={:}}

\usepackage[title]{appendix}
\renewcommand{\appendixname}{Appendix}

\usepackage[authoryear,comma]{natbib}
\renewcommand{\bibfont}{\small}
\setlength{\bibsep}{0em}

%\usepackage[sc,tiny,center]{titlesec}
\usepackage{titlesec}
\titleformat*{\section}{\sc \center}
\titleformat*{\subsection}{\it \center}
\renewcommand\thesection{\textnormal{{\Roman{section}}}}
\renewcommand{\refname}{Reference}
\usepackage[font={sc,small}]{caption}

\usepackage[%dvipdfmx,%
            bookmarks=true,%
            pdfstartview=FitH,%
            breaklinks=true,%
            colorlinks=true,%
            %allcolors=black,%
            citecolor=blue,
            linkcolor=red,
            pagebackref=true]{hyperref}

\renewcommand{\rmdefault}{ptm}
%\usepackage[lite]{mtpro2}
% use Palatinho-Roman as default font family
%\renewcommand{\rmdefault}{ppl}
\usepackage[scaled=0.88]{helvet}
\makeatletter   % Roman Numbers
\newcommand*{\rom}[1]{\expandafter\@slowromancap\romannumeral #1@}
\makeatother

\newcommand{\E}{\mathbb{E}}
\newcommand{\e}{\mathrm{e}}
\DeclareMathOperator*{\argmax}{argmax}
\renewcommand{\vec}[1]{\ensuremath{\mathbf{#1}}}
\newcommand{\gvec}[1]{{\boldsymbol{#1}}}

\usepackage[hmargin={1.2in,1.2in},vmargin={1.5in,1.5in}]{geometry}
\usepackage{threeparttable,booktabs,multirow} % This allows notes in tables

\usepackage{graphicx}
\usepackage{enumerate}
\usepackage{CJK}
\usepackage[title]{appendix}
\renewcommand{\appendixname}{Appendix}
\newtheorem{result}{Result}
\setlength{\unitlength}{1mm}

\topmargin -1cm        % read Lamport p.163
\oddsidemargin -0.04cm   % read Lamport p.163
\evensidemargin -0.04cm  % same as oddsidemargin but for left-hand pages
\textwidth 16.59cm
\textheight 21.94cm
\renewcommand\baselinestretch{1.15}
\parskip 0.25em
\parindent 1em
\linespread{1}

\newcommand{\code}{\texttt}
\newcommand{\bcode}[1]{\texttt{\blue{#1}}}
\newcommand{\rcode}[1]{\texttt{\red{#1}}}
\newcommand{\rtext}[1]{{\red{#1}}}
\newcommand{\btext}[1]{{\blue{#1}}}

% Set header and footer
%\usepackage{fancyhdr}
%\pagestyle{fancy}
%\fancyhead{}
%\fancyhead[LE,RO]{\thepage}
%\cfoot{}
%\renewcommand{\headrulewidth}{0pt}

%\pdfoptionpdfminorversion 6

\begin{document}

\title{\large{THE LETTER TO THE DATA EDITOR}}
\date{}
\maketitle

\vspace{-1.75cm}

\begin{flushleft}
Dear AEA Data Editor,
\end{flushleft}

This letter summarizes the changes that we made to the data repository (OPENICPSR-112005), as our responses to your comments on our submission to the \textit{American Economic Journal: Macroeconomics}. We sincerely appreciate the careful investigation of the replication team into our programs. We believe that we have fixed all the issues as per your request. Our responses are in the same order as the sections in your report.

\begin{enumerate}
    \item
    Summary
    \begin{itemize}
        \item
        We have added data citations for NGSPS, CNEC and SUSB to the article.
        \item
        We have removed all the files that were destined for the Data Editor only.
        \item
        We were a bit puzzled by this comment as we have filled all the OpenICPSR metadata fields. We think this may simply be a typo.
        \item
        The programs and instructions for appendix tables and figures had already been uploaded in our first submission. We apologize for any miscommunication. The complete code for inline numbers has been provided.
        %We have uploaded the Online Appendix to the data repository. The programs and instructions had already been uploaded in our first submission. We were not aware that the replication team cannot access the Appendix from the author center. We apologize for the miscommunication. Please do not hesitate to let us know if there is anything we did not upload.
    \end{itemize}
    \item
    Data Description
    \begin{itemize}
        \item
        NGSPS: Data citation has been added to the article. The source for ``The Regulations on the National General Survey of Pollution Sources" has been provided in the references.
        \item
        CNEC: Data citation has been added to the article. Additional information regarding the CNEC, including a landing page at the NBS of China, has been provided in the README (Section I.A).
        \item
        SUSB: Data citation has been added to the article.
    \end{itemize}
    \item
    Replication Steps:
    \begin{itemize}
        \item
        In the README, we did provide two tables (Tables 1 and 2) explaining which file depends on which dataset and generates which result. These two tables were mentioned in the opening paragraphs of Sections \rom{2} and \rom{3} of the README. To further highlight the two tables, we have added a paragraph at the end of Section \rom{1}.C to emphasize them.
    \end{itemize}    
    \item
    Findings:
    \begin{itemize}
        \item
        We have written two new MATLAB scripts \bcode{generate\_tables\_main.m} and \\
        \bcode{generate\_tables\_appendix.m} to directly compute and print Tables 4, 5, 6 and J.1 to MATLAB terminal. No manual calculation is needed anymore. We have also revised the instructions in the README accordingly.
        \item
        There are only two in-text numbers that require manual calculation: the average reduction in intensity for polluting industries and for the whole manufacturing sector (Section \rom{2} point 5 of the README). They are now automated in \bcode{Accounting.R} as well. The code and instruction are both updated accordingly.
    \end{itemize}
    \item
    MATLAB Terminal:
    \begin{itemize}
        \item
        The programs and instructions for appendix tables and figures had already been uploaded in our first submission.
        %Online Appendix has been uploaded to the OPENICPSR repository as well. All the programs have already been uploaded in our previous submission.
        \item
        We suspect that the reason that the data editor did not see the computation accuracy related information (optimality tolerance, etc.) is due to a different version of MATLAB being used. In the code, we have specified the option \bcode{'final-detailed'} when calling \bcode{fsolve} to request MATLAB to print information regarding computation accuracy. Such information was correctly printed when we tested our code in both MATLAB R2014a, R2016a and R2017b. We believe that minor revisions to the syntax should be sufficient for other versions.%We were not able to replicate the issue. We believe, however, that minor revisions to the syntax should be sufficient for other versions.
    \end{itemize}
\end{enumerate}



\end{document} 