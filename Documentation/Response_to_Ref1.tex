\documentclass[twoside,12pt,leqno]{article}
%\documentclass[twoside,12pt,leqno]{report}

\usepackage{amsmath,amssymb,amsfonts,amsthm,mathrsfs,multirow}
\usepackage[capposition=bottom]{floatrow} % For Figure Notes
\usepackage{graphicx,pstricks,epstopdf}
\usepackage{url}      % This package helps to typeset urls

%\usepackage{apacite}
%\usepackage{natbib}   % This is a great aid with bibliographies
%\setcitestyle{authoryear,round,semicolon,aysep={},yysep={,},notesep={:}}

\usepackage[title]{appendix}
\renewcommand{\appendixname}{Appendix}

\usepackage[authoryear,comma]{natbib}
\renewcommand{\bibfont}{\small}
\setlength{\bibsep}{0em}

%\usepackage[sc,tiny,center]{titlesec}
\usepackage{titlesec}
\titleformat*{\section}{\sc \center}
\titleformat*{\subsection}{\it \center}
\renewcommand\thesection{\textnormal{{\Roman{section}}}}
\renewcommand{\refname}{Reference}
\usepackage[font={sc,small}]{caption}

\usepackage[%dvipdfmx,%
            bookmarks=true,%
            pdfstartview=FitH,%
            breaklinks=true,%
            colorlinks=true,%
            %allcolors=black,%
            citecolor=blue,
            linkcolor=red,
            pagebackref=true]{hyperref}

\renewcommand{\rmdefault}{ptm}
%\usepackage[lite]{mtpro2}
% use Palatinho-Roman as default font family
%\renewcommand{\rmdefault}{ppl}
\usepackage[scaled=0.88]{helvet}
\makeatletter   % Roman Numbers
\newcommand*{\rom}[1]{\expandafter\@slowromancap\romannumeral #1@}
\makeatother

\newcommand{\E}{\mathbb{E}}
\newcommand{\e}{\mathrm{e}}
\DeclareMathOperator*{\argmax}{argmax}
\renewcommand{\vec}[1]{\ensuremath{\mathbf{#1}}}

\usepackage[hmargin={1.2in,1.2in},vmargin={1.5in,1.5in}]{geometry}
\usepackage{threeparttable,booktabs,multirow} % This allows notes in tables

\usepackage{graphicx}
\usepackage{enumerate}
\usepackage{CJK}

\usepackage[title]{appendix}
\renewcommand{\appendixname}{Appendix}
\newtheorem{result}{Result}
\setlength{\unitlength}{1mm}

\topmargin -1cm        % read Lamport p.163
\oddsidemargin -0.04cm   % read Lamport p.163
\evensidemargin -0.04cm  % same as oddsidemargin but for left-hand pages
\textwidth 16.59cm
\textheight 21.94cm
\renewcommand\baselinestretch{1.15}
\parskip 0.25em
\parindent 1em
\linespread{1}

% Set header and footer
%\usepackage{fancyhdr}
%\pagestyle{fancy}
%\fancyhead{}
%\fancyhead[LE,RO]{\thepage}
%\cfoot{}
%\renewcommand{\headrulewidth}{0pt}

%\pdfoptionpdfminorversion 6

\begin{document}

\title{\large{RESPONSES TO REFEREE 1'S COMMENTS}}
\date{}
\maketitle

\vspace{-1.75cm}

\begin{flushleft}
To Referee \#1,
\end{flushleft}

This letter summarizes our response to Referee 1��s report, which begins with ``The paper says no existing economic research...'' We thank Referee 1 again for the comments we received. Below, we respond to Referee 1's comments point by point.

\begin{enumerate}
    \item 
    We thank Referee 1 for mentioning the paper by \citet{Heetal:2019} to us. It is clear that the paper used a different dataset, which has a panel structure. However, since the authors did not provide enough details on the original source of the environmental dataset they used---possibly due to the lack of an official English translation for the name of the dataset---we had a really hard time identifying the exact dataset. Our educated guess is that they were using the firm-level data from \textit{Annual Statistic Report on Environment in China} (in Chinese \begin{CJK}{GBK}{kai}�й�����ͳ���걨\end{CJK}), but we cannot be 100\% sure.

    We talked with our colleagues who have direct access to the above dataset, and learned that there are at least two important differences between that dataset and the one we used. First, the firm-level emission data in the Annual Statistic Report were mostly self-reported by the firms, while the emission data in our dataset were collected by well-trained field staff who could also borrow manpower from local government agencies. As a result, the data quality of the Annual Statistic Report is likely to be lower. \citet{Heetal:2019} seemed to be concerned with the quality of their dataset as well, for which they wrote in their paper that ``Unlike the ASIF, however, we are less confident about the quality of ESR data.'' Second, while the treatment equipment data in our dataset plays a key role in our analysis, the Annual Statistic Report does not provide any information on the treatment equipment used by firms.

    \item 
    We have taken Referee 1's advice, removed some footnotes and put some into the main text. We have also removed italics from more than a dozen of words.

    Specifically, the following footnotes from our previous draft are removed: 2, 3, 4, 7, 19, 23, 28, and 41. The following footnotes are integrated into the main text: 10, 12, 22, 25, 36, 39, 44, and 49. We have also merged Footnotes 16 and 17. As a result, the total number of footnotes has been reduced from 52 to 35.

    \item 
    In our accounting exercise in Section I.C., we do take the U.S. as a benchmark, since we have data on the U.S. firm size distribution. In our quantitative analysis, however, we are not able to do that, simply because we do not have access to the U.S. firm-level data. Instead, we remove all the correlated distortions from our benchmark model. We did include some discussion on the economic interpretation of the measured wedges that we call distortions in Part \textit{Correlated Distortions} of Section II.A, with Appendix D.1 providing further details. We choose to call these measured wedges as distortions because we prefer to stay close to the literature. In that section, we have cited two survey articles by \citet{RestucciaRogerson:2013, RestucciaRogerson:2017} that elaborate this interpretation issue that the referee concerns. That said, we believe that it is of interest to carry out an additional exercise where the U.S. is the benchmark, if the U.S. firm-level data are available, and we leave this interesting exercise to future work.
\end{enumerate}

\bibliographystyle{aea}
\bibliography{./Dissertation1}

\end{document}
